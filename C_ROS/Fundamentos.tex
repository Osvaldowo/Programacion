\documentclass[12pt,a4paper]{article}

% Paquetes necesarios
\usepackage{hyperref}
\usepackage{amsmath, amssymb, amsthm}
\usepackage{enumitem}
\usepackage[spanish]{babel}
\usepackage[utf8]{inputenc}
\usepackage[T1]{fontenc}
\usepackage{fancyhdr}
\usepackage{geometry}
\usepackage{xcolor}
\usepackage{graphicx}
\usepackage{longtable} % Paquete para tablas largas
\usepackage{booktabs}   % Necesario para \toprule, \midrule, \bottomrule
\usepackage{array}      % Necesario para >{\raggedright\arraybackslash}

% --- GEOMETRÍA Y ENCABEZADOS ---
\geometry{
  margin=2.5cm,
  top=3.5cm,
  headheight=100pt,
  headsep=25pt
}

\definecolor{vino}{RGB}{128,0,64}

\pagestyle{fancy}
\fancyhf{}

\fancyheadoffset[L]{0pt}
\fancyheadoffset[R]{0pt}

\fancyhead[L]{\includegraphics[height=1.2cm]{Imagenes/LogoIPN.png}}
\fancyhead[C]{\small Instituto Politécnico Nacional \\
Unidad Profesional Interdisciplinaria en Ingeniería y Tecnologías Avanzadas \\
Análisis y Síntesis de Mecanismos}
\fancyhead[R]{\includegraphics[height=1.2cm]{Imagenes/Logo_upiita.png}}

\fancyfoot[C]{\small Flores Oropeza Osvaldo \hspace{1cm} C-ROS \hspace{1cm} \thepage}

\renewcommand{\headrulewidth}{0.4pt}
\renewcommand{\footrulewidth}{0.4pt}
\renewcommand{\headrule}{\hbox to\headwidth{\color{vino}\leaders\hrule height \headrulewidth\hfill}}
\renewcommand{\footrule}{\hbox to\headwidth{\color{vino}\leaders\hrule height \footrulewidth\hfill}}

\setlength{\parindent}{0pt}

% --- INICIO DEL DOCUMENTO ---
\begin{document}

\begin{center}
    {\Large \textbf{Fundamentos de Cinemática y Dinámica para el Análisis de Mecanismos y Robótica}} \\[0.3cm]
\end{center}

\vspace{0.5cm}

\section*{Introducción}

El diseño y análisis de sistemas mecánicos, desde simples herramientas manuales hasta complejos robots industriales, se fundamenta en un conjunto de principios básicos que rigen el movimiento y las fuerzas. La comprensión profunda de estos conceptos es esencial para el ingeniero, ya que constituyen el lenguaje con el que se conciben, analizan y optimizan las máquinas que definen el mundo moderno.

Este informe presenta una investigación exhaustiva sobre los conceptos fundamentales de la teoría de mecanismos y la robótica, basándose en una revisión de la literatura técnica especializada \cite{lopez-cajun-2008}. La investigación se estructura en dos partes. La primera parte ofrece un análisis técnico detallado de cada concepto, estableciendo definiciones precisas, comparando las perspectivas de diferentes autores y explorando las implicaciones teóricas y prácticas de cada idea. Cada afirmación y dato se atribuye rigurosamente a su fuente original para garantizar la trazabilidad y la precisión académica. La segunda parte consiste en un resumen conceptual diseñado para ser accesible a un público no especializado, traduciendo la terminología técnica a un lenguaje claro y comprensible, con el objetivo de proporcionar una introducción intuitiva al fascinante campo del análisis de mecanismos.

% --- INICIO DE LA TABLA LARGA ---
\begin{longtable}{>{\raggedright\arraybackslash}p{2.5cm} >{\raggedright\arraybackslash}p{5cm} >{\raggedright\arraybackslash}p{5cm}}

\caption{Comparativa de Definiciones Fundamentales en la Teoría de Mecanismos}
\label{tab:comparativa} \\
\toprule
\textbf{Concepto} & \textbf{Definición según López Cajún \& Ceccarelli} & \textbf{Definición según Hernández} \\
\midrule
\endfirsthead

\caption{Continuación de la tabla \ref{tab:comparativa}} \\
\toprule
\textbf{Concepto} & \textbf{Definición según López Cajún \& Ceccarelli} & \textbf{Definición según Hernández} \\
\midrule
\endhead

\midrule
\multicolumn{3}{r}{\textit{Continúa en la siguiente página...}} \\
\endfoot

\bottomrule
\endlastfoot

Mecanismo & Un sistema de cuerpos diseñados para convertir movimientos de (y/o fuerzas sobre) uno o varios cuerpos en movimientos restringidos de (y/o fuerzas sobre) otro cuerpo \cite{lopez-cajun-2008}. & Dispositivo cuyo análisis se centra en el concepto de movimiento (posiciones, velocidades, aceleraciones). \\
\addlinespace
Máquina & Un sistema mecánico que realiza una tarea específica, transfiere potencia, transmite fuerzas y/o transforma movimiento \cite{lopez-cajun-2008}. & Dispositivo que incluye la idea de transmisión de fuerzas, predominando esta sobre la de movimiento. \\
\addlinespace
Cinemática & Rama de la Mecánica Aplicada que estudia la geometría del movimiento sin importar las causas que lo producen \cite{lopez-cajun-2008}. & Estudia el movimiento de los mecanismos sin tener en cuenta las causas que lo producen, dependiendo directamente de las características geométricas. \\
\addlinespace
Dinámica & (No definido explícitamente, pero implícito en el estudio de las "fuerzas") \cite{lopez-cajun-2008}. & Incluye en su estudio las acciones que provocan el movimiento de los mecanismos, estableciendo la relación causa-efecto. \\
\addlinespace
Cuerpo Rígido & Un sistema de puntos materiales tal que la distancia entre dos puntos cualesquiera del mismo permanece constante durante el movimiento \cite{lopez-cajun-2008}. & Una idealización donde se asume que los sólidos no se deforman. Es una ventaja porque las ecuaciones que gobiernan el problema son mucho más sencillas. \\
\addlinespace
Grado de Libertad & El número de variables necesario y suficiente que define de forma única la posición y la orientación de todos los eslabones de la cadena \cite{lopez-cajun-2008}. & (Implícito en la fórmula de Grübler) El número de entradas independientes necesarias para definir la configuración del mecanismo. \\

\end{longtable}
% --- FIN DE LA TABLA LARGA ---

\section{Parte I: Investigación Detallada de Conceptos Fundamentales}

\subsection{Las Entidades del Movimiento: De lo Abstracto a lo Físico}

\subsubsection{La Partícula: Un Modelo Idealizado Ausente}
En la física fundamental, el concepto de \textit{partícula} es el punto de partida para el estudio del movimiento. Sin embargo, en el análisis de mecanismos este término está ausente, ya que una partícula no puede rotar ni tener orientación, siendo un modelo insuficiente para los componentes de una máquina \cite{lopez-cajun-2008}.

\subsubsection{El Cuerpo Rígido: La Base del Análisis de Mecanismos}
El \textit{cuerpo rígido} es la piedra angular sobre la que se construye toda la cinemática de mecanismos. Se define como un sistema de puntos donde la distancia entre dos puntos cualesquiera permanece constante durante el movimiento \cite{lopez-cajun-2008}. Es una idealización, pero válida, ya que los eslabones se diseñan para que sus deformaciones sean mínimas.

\subsection{El Estudio del Movimiento: Cinemática y Dinámica}

\subsubsection{Cinemática: La Geometría del Movimiento}
La \textit{Cinemática} es la rama de la mecánica que se ocupa de la "geometría del movimiento" \cite{lopez-cajun-2008}. Su objetivo es describir el movimiento de los cuerpos sin considerar las fuerzas que lo causan, determinando posición, velocidad y aceleración.

\subsubsection{Dinámica: Las Fuerzas Detrás del Movimiento}
La \textit{Dinámica} incluye en su estudio las acciones (fuerzas y momentos) que provocan o modifican el movimiento, estableciendo una relación directa de causa y efecto. Introduce los conceptos de masa, peso e inercia \cite{lopez-cajun-2008}.

\subsection{Anatomía de los Sistemas Mecánicos}

\subsubsection{Distinción entre Sistema Mecánico, Mecanismo y Máquina}
Una \textbf{máquina} es un sistema mecánico que realiza una tarea específica, como transferir potencia o transmitir fuerzas. Un \textbf{mecanismo} es un sistema de cuerpos diseñado para convertir movimientos y/o fuerzas; es el "corazón" de una máquina \cite{lopez-cajun-2008}.

\subsubsection{Eslabón (Link)}
El \textit{eslabón} es el componente fundamental de un mecanismo, definido como cada una de las partes individuales, consideradas rígidas, que lo componen \cite{lopez-cajun-2008}.

\subsubsection{Estructura}
Una \textit{estructura} es un ensamblaje de eslabones y uniones incapaz de moverse, es decir, un "mecanismo bloqueado" con cero grados de libertad \cite{lopez-cajun-2008}.

\subsection{Caracterización de la Movilidad y el Espacio de Operación}

\subsubsection{Grado de Libertad (GDL) y la Ecuación de Grübler}
El \textit{grado de libertad (GDL)} es el número de entradas independientes requeridas para definir la posición de todos los eslabones \cite{lopez-cajun-2008}. Para mecanismos planos, se calcula con la \textbf{ecuación de Grübler} \cite{lopez-cajun-2008}:
\[ M = 3(n-1) - 2j_p - j_h \]
o en su forma simplificada para pares de 1 GDL:
\[ l = 3(n-1) - 2p_i \]
donde $n$ es el número de eslabones, $j_p$ o $p_i$ es el número de pares de 1 GDL y $j_h$ es el número de pares de 2 GDL.

\subsection{Un Universo de Soluciones: Tipología de Mecanismos}
Los mecanismos se pueden clasificar en varios grupos principales, como mecanismos de barras, de levas, de engranajes, de tornillo y con elementos flexibles \cite{lopez-cajun-2008}.

\subsection{Potencia y Control: Sistemas de Accionamiento y Transmisión}

\subsubsection{Actuadores: Los Músculos de las Máquinas}
Los \textit{actuadores} son los dispositivos que suministran la energía y el movimiento de entrada a un mecanismo. Los tipos principales son eléctricos, hidráulicos y neumáticos \cite{lopez-cajun-2008}.

\subsubsection{Transmisores y Reductores: Modulando la Potencia y el Movimiento}
Las \textit{transmisiones mecánicas} (engranajes, correas) conectan el actuador al mecanismo para adaptar la velocidad y el par. Los \textit{reductores} disminuyen la velocidad para aumentar el par \cite{lopez-cajun-2008}.

\subsection{La Evolución de los Mecanismos: Introducción a la Robótica}

\subsubsection{El Robot: Un Sistema Mecánico Programable}
Un \textit{robot} es un mecanismo programable, típicamente de cadena cinemática abierta y con varios grados de libertad, integrado con un sistema de control y sensores. La palabra "robot" deriva de la palabra checa "robota", que significa "trabajo forzado" \cite{lopez-cajun-2008}.

\subsubsection{El Manipulador: El Brazo del Robot}
El \textit{manipulador} es la estructura mecánica del robot (el "brazo"), mientras que el \textit{robot} es el sistema completo, que integra el manipulador, los actuadores, los sensores y el sistema de control \cite{lopez-cajun-2008}.

\section{Parte II: Resumen Introductorio al Análisis de Mecanismos}
\subsection{¿De qué están hechas las máquinas? Los ladrillos del movimiento}
Imagine el esqueleto humano. Está formado por huesos rígidos (como el fémur o el húmero) conectados por articulaciones (la rodilla, el codo) que permiten el movimiento. Las máquinas y los mecanismos se construyen de manera muy similar.
Los "huesos" de una máquina se llaman eslabones. Son las piezas rígidas e individuales que, al unirse, forman la estructura de la máquina. Un eslabón puede ser una barra, una palanca, un engranaje o cualquier pieza que no se doble ni se estire significativamente mientras trabaja. La idea de que son perfectamente rígidos es una simplificación muy útil llamada la suposición de cuerpo rígido, que nos permite estudiar su movimiento de forma mucho más sencilla.
Las "articulaciones" que conectan estos eslabones se llaman pares cinemáticos o simplemente uniones. Permiten que los eslabones se muevan unos respecto a otros de formas muy específicas. Las más comunes son las uniones de pasador (como la bisagra de una puerta, que solo permite girar) y las uniones deslizantes (como un cajón que solo puede moverse hacia adentro y hacia afuera).
Un conjunto de eslabones conectados por uniones forma un mecanismo. Si este mecanismo está diseñado para no moverse en absoluto, como la estructura de un puente, lo llamamos una estructura.

\subsection{¿Cómo se mueven? La Cinemática, o la geometría en acción}
Una vez que tenemos nuestro "esqueleto" mecánico, la siguiente pregunta es: ¿cómo se mueve? El estudio de esta pregunta es la cinemática. La cinemática es la geometría del movimiento. No le importa por qué se mueven las cosas (las fuerzas), solo le interesa describir cómo lo hacen: qué camino siguen, a qué velocidad van y cómo aceleran.
Para describir el movimiento, los ingenieros utilizan un concepto clave: el grado de libertad (GDL). Esto suena complicado, pero es una idea simple: es el número de "mandos" o "controles" independientes que necesitas para que la máquina haga exactamente lo que quieres.
\begin{itemize}
    \item Un mecanismo con un grado de libertad es el más común. Piensa en el limpiaparabrisas de un coche. Un solo motor hace que todo el sistema de brazos se mueva de una manera perfectamente predecible. Solo necesitas un "mando" (el motor).
    \item Un sistema con cero grados de libertad es una estructura rígida. No se mueve.
    \item Un sistema con más de un grado de libertad necesita varios "mandos" trabajando juntos. Un brazo robot es un buen ejemplo: tiene múltiples motores en el hombro, el codo y la muñeca, y todos deben coordinarse para colocar la mano en un punto exacto del espacio.
\end{itemize}

\subsection{¿Por qué se mueven? La Dinámica, la fuerza detrás del movimiento}
Si la cinemática es el "cómo", la dinámica es el "por qué". La dinámica introduce las fuerzas en la ecuación. Estudia cómo las fuerzas (como la de un motor o la de la gravedad) hacen que los eslabones se muevan y aceleren.
Cuando un eslabón con masa acelera, genera una "resistencia" a ese cambio de movimiento. Esta resistencia es una fuerza real llamada fuerza de inercia. En una máquina que se mueve muy rápido, como un motor de carreras, estas fuerzas de inercia pueden ser enormes, mucho más grandes que el peso de las propias piezas. Por eso, el análisis dinámico es crucial para diseñar máquinas que no se rompan bajo su propio movimiento.

\subsection{Construyendo la máquina: Actuadores y Transmisiones}
Para que un mecanismo se mueva, necesita "músculos". Estos músculos se llaman actuadores. Son los motores, cilindros hidráulicos o neumáticos que proporcionan la energía para iniciar el movimiento. La elección del actuador correcto depende de si se necesita mucha fuerza (hidráulico), mucha precisión (eléctrico) o un movimiento rápido y barato (neumático).
A veces, la velocidad y la fuerza que proporciona un motor no son las adecuadas para la tarea. Por ejemplo, un motor eléctrico funciona mejor a alta velocidad, pero una prensa necesita moverse lentamente y con mucha fuerza. Aquí es donde entran las transmisiones, como las cajas de cambios de un coche. Una transmisión, a menudo usando engranajes, actúa como un intermediario que puede cambiar la alta velocidad y baja fuerza del motor por la baja velocidad y alta fuerza que necesita el mecanismo. Si la transmisión reduce la velocidad, se llama reductor.

\subsection{El siguiente nivel: Robots y Manipuladores}
¿Qué es un robot? En esencia, es un mecanismo muy avanzado. Generalmente, es un "brazo" (llamado manipulador) con muchos grados de libertad, lo que le da una gran flexibilidad para moverse. Pero lo que realmente define a un robot es que es programable. Está conectado a un "cerebro" (un ordenador) y a "sentidos" (sensores) que le permiten realizar una gran variedad de tareas diferentes, desde soldar la carrocería de un coche hasta realizar una cirugía delicada.
Así, el manipulador es solo el cuerpo mecánico del robot, mientras que el robot es el sistema completo: cuerpo, músculos, sentidos y cerebro, todo trabajando en conjunto. Este es el punto donde la ingeniería mecánica se une con la electrónica y la informática para crear las máquinas inteligentes del presente y del futuro.

\bibliographystyle{IEEEtran} 

\bibliography{Referencias.bib}   


\end{document}
