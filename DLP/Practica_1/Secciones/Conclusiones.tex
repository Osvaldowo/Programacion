%! TEX root = C:\Users\osval\OneDrive - Instituto Politecnico Nacional\Codigos\Programacion\DLP\Practica_1\main.tex

\subsection*{Flores Oropeza Osvaldo}
    Durante la práctica note lo complejo que es diseñar etapas de potencia, ya sean de alimentación o de audio. Igualmente entendí la importancia de preparar con antelación los módulos necesarios para el correcto desarrollo de la práctica. \\
    En la primera actividad tuve problemas al darle significado físico a las salidas del sistema, esto debido a que lo ideal es que las salidas sean consecuencias del sistema diseñado. Esto es muy útil a la hora de diseñar sistemas de control, lo que nos ayudará durante la carrera. \\
    \textcolor{Amarillo}{\textbf{Frase:}} El \textcolor{Amarillo}{esfuerzo} sin objetividad y realidad no sirve. \\
    \textcolor{Amarillo}{\textbf{Metáfora:}} La metáfora habla de aceptación, cada persona es quien es no debería intentar ser alguien más, pero si debería intentar se mejor.

\subsection*{Ramírez Aguilar Víctor Saul}
    En esta practica se vio a simplicidad que puede llegar a tener el realizar algunas funciones basadas en circuitos de lógica secuencial y combinatoria con el uso de lenguajes VHDL y verilog para la programación de tarjetas FPGA, ya que no es muy diferente a otros lenguajes de programación donde podemos tener funciones para realizar tareas especificas, o el uso de vectores para el uso de lineas de datos como fue el caso del registro de 8 bits donde la función se realizo en pocas lineas de código lo que en un diseño con lógica combinacional física pudo haber requerido el uso de mas componentes. \\
    \textcolor{Amarillo}{\textbf{Frase:}} El \textcolor{Amarillo}{esfuerzo} trae buenas recompensas\\
    \textcolor{Amarillo}{\textbf{Lectura:}} Al ser estudiante de mecatronica creo que lo que quisiera transmitir al mundo seria compartir mi capacidad para tratar de resolver problemas de la vida cotidiana con soluciones tecnológicas.   
    
\subsection*{Reyes Lira Alejando}

    La práctica evidenció la capacidad y flexibilidad de los PLD para diseñar sistemas digitales diversos. Se inició con la descripción de una tabla de verdad en HDL, mostrando la eficiencia en la abstracción de lógica combinacional. Luego, se implementó un registro de 8 bits con flip-flops tipo D, destacando su importancia en el almacenamiento de datos. Posteriormente, se desarrolló un contador con salida a display de 7 segmentos, integrando conteo, decodificación e interfaz visual, esenciales en instrumentación. Finalmente, se aplicó el control de señales con la generación de sonidos de sirena, demostrando la interacción directa del PLD con el entorno. En conjunto, la práctica consolidó cómo estos dispositivos abarcan desde almacenamiento hasta generación de señales dinámicas, reafirmando su papel central en la tecnología digital.\\
    \textcolor{Amarillo}{\textbf{Frase:}} Enfoca tu \textcolor{Amarillo}{energia}, no en luchar contra lo que fuiste, sino en construir la persona que quieres llegar a ser.\\
    \textcolor{Amarillo}{\textbf{Metáfora:}} la metáfora nos enseña que la felicidad no proviene de lograr ser una buena imitación de otros, sino de un acto de valentía: mirar hacia nuestro interior, descubrir nuestra naturaleza única y abrazar nuestro propio y auténtico propósito, que es tan valioso como cualquier otro. 