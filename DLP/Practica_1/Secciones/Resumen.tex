%! TEX root = C:\Users\osval\OneDrive - Instituto Politecnico Nacional\Codigos\Programacion\DLP\Practica_1\main.tex

La práctica consiste en el uso de lenguajes de descripción de hardware (VHDL y Verilog) para implementar y simular circuitos digitales en una tarjeta de desarrollo FPGA. Se trabajan tanto circuitos combinacionales como secuenciales: funciones lógicas básicas, un registro de 8 bits con flip-flops tipo D, un contador binario ascendente/descendente de 4 bits con salida a display de 7 segmentos, y generadores de sonido. Finalmente, se integran los distintos módulos en un Top Level Design (TLD), que concentra todas las funciones en un solo proyecto. El objetivo es que el estudiante se familiarice con el proceso completo de diseño digital: programación, simulación, implementación y prueba en hardware real.