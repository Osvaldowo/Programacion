%! TEX root = C:\Users\osval\OneDrive - Instituto Politecnico Nacional\Codigos\Programacion\DLP\Practica_1\main.tex

A prática consiste no uso de linguagens de descrição de hardware (VHDL e Verilog) para implementar e simular circuitos digitais em uma placa de desenvolvimento FPGA. São trabalhados circuitos combinacionais e sequenciais: funções lógicas básicas, um registrador de 8 bits com flip-flops tipo D, um contador binário crescente/decrescente de 4 bits com saída em display de 7 segmentos e geradores de som. Por fim, todos os módulos são integrados em um Top Level Design (TLD), que reúne todas as funções em um único projeto. O objetivo é que o aluno se familiarize com todo o processo de projeto digital: programação, simulação, implementação e teste em hardware real.