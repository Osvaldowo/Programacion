%! TEX root = C:\Users\osval\OneDrive - Instituto Politecnico Nacional\Codigos\Programacion\DLP\Practica_1\main.tex

\newpage
\textbf{Tabla de Contenido.}

\begin{table}[H]
    \centering
    % Table generated by Excel2LaTeX from sheet 'Hoja1'
\begin{tabular}{|c|c|c|c|c|}
\hline
Puntos & Codigo VHDL & Codigo Verilog & Fotos  & Simulación \bigstrut\\
\hline
1. Funciones &   &   &   &  \bigstrut\\
\hline
2. Registro 8 bits &   &   &   &  \bigstrut\\
\hline
3. Contador &   &   &   &  \bigstrut\\
\hline
4. Sirenas & \cellcolor[rgb]{ .816,  .816,  .816} &   &   & \cellcolor[rgb]{ .816,  .816,  .816} \bigstrut\\
\hline
5. TLD  &   &   &   & \cellcolor[rgb]{ .816,  .816,  .816} \bigstrut\\
\hline
\multicolumn{1}{|p{7.665em}|}{6. Encoder, Pmod, RS232} &   & \cellcolor[rgb]{ .816,  .816,  .816} &   & \cellcolor[rgb]{ .816,  .816,  .816} \bigstrut\\
\hline
CH & \cellcolor[rgb]{ .816,  .816,  .816} & \cellcolor[rgb]{ .816,  .816,  .816} & \cellcolor[rgb]{ .816,  .816,  .816} & \cellcolor[rgb]{ .816,  .816,  .816} \bigstrut\\
\hline
\end{tabular}%
    \caption{Tabla de contenido}
    \label{tab:Tabla de contenido}
\end{table}